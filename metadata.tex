%%
%% This is file `example/metadata.tex',
%% generated with the docstrip utility.
%%
%% The original source files were:
%%
%% install/buptgraduatethesis.dtx  (with options: `metadata')
%% 
%% This file is a part of the example of BUPTGraduateThesis.
%% 

%% 涉密论文保密年限
\classdur{三年}

%% 学号
\studentid{2019110193}

%% 论文题目
\ctitle{面向时延保障的段路由航点选择算法研究}
\etitle{Research on Segment Routing List Generate Algorithm for Network Latency}

%% 申请学位
\cdegree{工学硕士}

%% 院系名称
\cdepartment{信息与通信工程学院}
\edepartment{School of Information and Communication Engineering}

%% 专业名称
\cmajor{信息与通信工程}
\emajor{Information and Communication Engineering}

%% 你的姓名
\cauthor{赖丽蓉}
\eauthor{Lirong Lai}

%% 博士后研究工作报告-分类号
\classnumber{O441.3}

%% 博士后研究工作报告-UDC
\udc{621.396.9}

%% 博士后研究工作报告-学校编号
\schoolserial{147227}

%% 博士后研究工作起始时间
\startdate{2014年10月29日}

%% 博士后研究工作期满时间
\finishdate{2016年4月2日}

%% 你导师的姓名
\csupervisor{刘韵洁}
\esupervisor{Yunjie Liu}

%% 日期自动生成,也可以取消注释下面一行,自行指定日期

%% 中文摘要
\cabstract{%
段路由(Segment Routing, SR)是一种网络层的流量调度思想,核心在于用分段的路由目标取代源路由中的逐跳的目的标签,这种路径编码的方式可以应用于流量调度、性能测量、故障恢复等多种方向。随着网络业务对时延提出了越来越高的需求,网络需要从各个层面考虑时延保障的效果,诞生于网络层的段路由技术就非常适合用于时延保障。

段路由已经广泛应用于网络流量调度。段路由思想下的流量调度方案使用段路由策略指导报文转发,其在数据包层面表现为段路由航点列表,即段路由的任何策略最后都会落实为段路由航点列表。但是目前针对段路由航点列表生成算法的研究目标往往集中在优化带宽使用效率和提高全网吞吐量,导致网络层缺少时延保障的能力。针对这一问题,本文基于段路由的架构和设计思想,分别从集中式控制角度和分布式控制角度提出了两种面向时延保障的段路由航点列表生成算法。

首先,本文基于段路由的集中式控制方式提出第一种段路由航点生成算法。该算法通过软件定义网络的全局视角,为每一条链路建立考虑排队时延的链路权重,并在拓扑抽象的基础上进一步通过辅助图对网络拓扑的数据模型进行降维,在降维后的数据模型上应用贝尔曼-福特算法计算段路由航点列表。整个算法以降低段路由流量端到端时延和降低段路由航点列表算法时间复杂度为主要优化目标,最终得出具有时延保障效果的段路由航点列表。

然后,本文基于段路由的分布式控制方式提出第二种段路由航点生成算法。该算法首先设计了用于规划段路由节点时延探测的对象的相邻段路由节点发现算法;其次在每个段路由节点上生成差分时延矩阵来记录流量调度目标时延,并设计时延矩阵的更新算法保障网络时延信息可以被全网段路由节点以分布式方法获取并更新到路由节点自身的时延矩阵里;最后设计了段路由航点列表生成算法使段路由入口节点可以独立计算出具有指定时延保障效果的段路由航点列表。

最后,本文基于可编程 P4 软件交换机 BMv2 实现上述两种算法,并进行组网实验验证。将集中式段路由航点生成算法的对照组设置为仅用带宽计算段路由航点列表的算法,通过实验验证得到:在数据中心网络和运营商网络中使用本文提出的集中式算法调度流量所产生的时延,比对照组分别降低了8.22\%和3.34\%;将分布式段路由航点生成算法的对照组设置为分布式最短路径算法,通过实验验证得到:在损失一定网络吞吐量的情况下,在数据中心网络和运营商网络中用本文提出的分布式算法调度流量所产生的时延,比对照组分别降低了53.3\%和65.6\%。实验结果表明,基于时延信息进行段路由航点列表的计算可以降低流量端到端时延,为时延敏感业务提供网络层的技术支持。
}

%% 中文关键词,关键词之间用 \kwsep 分割
\ckeywords{软件定义网络 \kwsep 段路由 \kwsep 流量工程 \kwsep 网络时延}

%% 英文摘要
\eabstract{%
Segment Routing (SR) is a network layer traffic engineering idea, the core of which is to replace the hop-by-hop destination label in the source route with a segmented routing destination, this path coding approach can be applied to traffic engineering, performance evaluation, failure recovery, and other directions. As network services place increasingly high demands on latency, the network needs to consider the effects of latency guarantees at all levels, and SR which is born at the network layer is ideally suited for latency guarantees.

SR has been widely used in network traffic engineering. The SR traffic engineering uses SR policies to guide the forwarding of packets. The policy is expressed as a list of SR transpoints in the packet header, which means any policy of SR will eventually be implemented as a transpoint list. However, the current research objectives of SR transpoint list generation algorithms are often focused on optimizing bandwidth usage efficiency or improving network throughput, resulting in a lack of latency guarantee. To address this problem, this paper proposes two latency-guaranteed SR transpoint list generation algorithms based on the architecture of SR, from the perspective of centralized control and distributed control respectively.

Firstly, the first SR transpoint generation algorithm is proposed based on the centralized control of SR. The algorithm establishes link weights for each link considering queuing latency through the global perspective of software-defined networking, further reduces the dimensionality of the data model of the network topology through auxiliary graphs based on topology abstraction, and applies the Bellman-Ford algorithm to calculate the SR transpoint list on the reduced data model. The whole algorithm is optimized with the main objectives of reducing the end-to-end latency of SR traffic and reducing the time complexity of the SR transpoint list generation algorithm and finally arrives at an SR transpoint list with a latency guarantee effect.

Then, this paper proposes a second SR transpoint generation algorithm based on the distributed control of SR. The algorithm firstly designs an adjacent SR node discovery algorithm for planning the object of SR node latency detection. Secondly, the algorithm generates a latency matrix at each SR node to record the target latency of traffic engineering and designs a latency matrix update algorithm to ensure that the latency information can be obtained by every SR node in a distributed way and updated in its latency matrix. Finally, it designs an SR transpoint list generation algorithm so that the entry SR node can independently compute an SR transpoint list with the specified latency guarantee effect.

Finally, this paper implements the above two algorithms based on P4 and BMv2 and performs network experiments to verify them. The control group of the centralized SR transpoint generation algorithm is set to the algorithm that only uses bandwidth to calculate the SR transpoint list, and it is experimentally verified that the latency generated by scheduling traffic using the centralized algorithm proposed in this paper is 8.22\% and 3.34\% lower than that of the control group in the data center network and the ISP network respectively. The control group of the distributed SR transpoint generation algorithm is set to the shortest path algorithm, and it is experimentally verified that the latency generated by scheduling traffic with the distributed algorithm proposed in this paper in the data center network and the ISP network is reduced by 53.3\% and 65.6\%, respectively, compared with the control group under the loss of certain network throughput. The experimental results show that the calculation of SR transpoint lists based on latency information can reduce traffic end-to-end latency and provide network technical support for latency-sensitive services.
}

%% 英文关键词,也用 \kwsep 分割
\ekeywords{%
Software-Defined Networking \kwsep Segment Routing \kwsep Traffic Engineering \kwsep Network Latency}
