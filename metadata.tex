%%
%% This is file `example/metadata.tex',
%% generated with the docstrip utility.
%%
%% The original source files were:
%%
%% install/buptgraduatethesis.dtx  (with options: `metadata')
%% 
%% This file is a part of the example of BUPTGraduateThesis.
%% 

%% 涉密论文保密年限
\classdur{三年}

%% 学号
\studentid{2019110193}

%% 论文题目
\ctitle{面向时延保障的段路由航点选择算法研究}
\etitle{Research on Segment Routing List Generate Algorithm for Network Latency}

%% 申请学位
\cdegree{工学硕士}

%% 院系名称
\cdepartment{信息与通信工程学院}
\edepartment{School of Information and Communication Engineering}

%% 专业名称
\cmajor{信息与通信工程}
\emajor{Information and Communication Engineering}

%% 你的姓名
\cauthor{赖丽蓉}
\eauthor{Lirong Lai}

%% 博士后研究工作报告-分类号
\classnumber{O441.3}

%% 博士后研究工作报告-UDC
\udc{621.396.9}

%% 博士后研究工作报告-学校编号
\schoolserial{147227}

%% 博士后研究工作起始时间
\startdate{2014年10月29日}

%% 博士后研究工作期满时间
\finishdate{2016年4月2日}

%% 你导师的姓名
\csupervisor{刘韵洁}
\esupervisor{Yunjie Liu}

%% 日期自动生成,也可以取消注释下面一行,自行指定日期

%% 中文摘要
\cabstract{%
段路由是一种网络层的路由思想,正在越来越多地应用于网络流量调度中。而随着网络中的业务对时延提出了越来越高的要求,网络需要从各个层面考虑时延保障的效果。本文提出了两种算法,第一种段路由航点生成算法将通过基于软件定义网络控制器的集中式算法实现一种网络拓扑层的段路由策略计算,并得出具有一定程度延迟保证效果的段列表,并通过实验验证得到该算法比只使用带宽信息来生成段路由航点列表对服务请求的时延保障效果好了14.7\%。第二个航点路由算法则是以分布式的为主导思想,通过规划节点时延探测的对象、生成基于差分思想分组的时延矩阵、时延矩阵的更新方式以及段路由航点列表的生成方式最终得到具有时延保障效果的段路由列表,并且经过实验验证得出结论,在损失一定网络吞吐量的情况下该算法在保障时延需求方面比使用传统最短路径算法的协议好43.9\%。
}

%% 中文关键词,关键词之间用 \kwsep 分割
\ckeywords{软件定义网络 \kwsep 段路由 \kwsep 流量工程 \kwsep 网络时延}

%% 英文摘要
\eabstract{%
Segment Routing is a network layer routing idea that is increasingly being used in network traffic engineering. And as the services in the network put higher and higher demands on the latency, the network needs to consider the effect of latency guarantee from all levels. In this paper, we propose two algorithms, the first of which is a Segment List generation algorithm that uses a centralized algorithm based on a Software Defined Network controller to calculate a Segment Routing policy at the network topology level and derive a Segment List with a certain degree of latency guarantee. The algorithm is experimentally validated to be 14.7\% better than using only bandwidth information to generate a Segment List to guarantee the latency of requests. The second Segment Routing algorithm is based on a distributed approach, which plans the nodes' latency detection objects, generates a latency matrix based on differential grouping, updates the delay matrix, and generates a Segment List to obtain a latency-safe request. The experimental verification concluded that the algorithm is 43.9\% better than the protocol using the traditional shortest path algorithm in terms of securing the delay requirements at a certain loss of network throughput.
}

%% 英文关键词,也用 \kwsep 分割
\ekeywords{%
Software Defined Network \kwsep Segment Routing \kwsep Traffic Engineering \kwsep Network Latency}
