%%
%% This is file `example/metadata.tex',
%% generated with the docstrip utility.
%%
%% The original source files were:
%%
%% install/buptgraduatethesis.dtx  (with options: `metadata')
%% 
%% This file is a part of the example of BUPTGraduateThesis.
%% 

%% 涉密论文保密年限
\classdur{三年}

%% 学号
\studentid{2019110193}

%% 论文题目
\ctitle{面向时延保障的段路由航点选择算法研究}
\etitle{Research on Segment Routing List Generate Algorithm for Network Latency}

%% 申请学位
\cdegree{工学硕士}

%% 院系名称
\cdepartment{信息与通信工程学院}
\edepartment{School of Information and Communication Engineering}

%% 专业名称
\cmajor{信息与通信工程}
\emajor{Information and Communication Engineering}

%% 你的姓名
\cauthor{赖丽蓉}
\eauthor{Lirong Lai}

%% 博士后研究工作报告-分类号
\classnumber{O441.3}

%% 博士后研究工作报告-UDC
\udc{621.396.9}

%% 博士后研究工作报告-学校编号
\schoolserial{147227}

%% 博士后研究工作起始时间
\startdate{2014年10月29日}

%% 博士后研究工作期满时间
\finishdate{2016年4月2日}

%% 你导师的姓名
\csupervisor{刘韵洁}
\esupervisor{Yunjie Liu}

%% 日期自动生成,也可以取消注释下面一行,自行指定日期

%% 中文摘要
\cabstract{%
段路由(Segment Routing, SR)是一种网络层的流量调度思想,核心在于用分段的路由目标取代源路由中的逐跳的目的标签,这种路径编码的方式可以应用于网络编程。段路由已经广泛地应用于网络流量调度中,而随着网络中的业务对时延提出了越来越高的要求,网络需要从各个层面考虑时延保障的效果,诞生于网络层的段路由技术就非常适合用于时延保障。

段路由策略可以指导报文转发,其在数据包层面表现为段路由航点列表,即段路由的任何策略最后都会落实在段路由航点列表。但是目前针对段路由航点列表生成的研究目标往往集中在优化带宽使用效率和提高全网吞吐量,导致网络层缺少时延保障的能力。针对这一问题,本文基于段路由的架构和设计思想,分别从集中式控制角度和分布式控制角度提出了两种段路由航点列表生成算法。

首先,本文基于段路由的集中式控制方式提出第一种段路由航点生成算法。该算法通过软件定义网络的全局视角,在传统拓扑抽象的基础上进一步对数据模型进行降维,以最短流量端到端时延和航点列表计算时间复杂度为主要优化目标,最终得出具有延迟保证效果的段路由航点列表。

然后,本文基于段路由的分布式控制方式提出第二种段路由航点生成算法。该算法首先通过设计了相邻段路由节点发现算法,用于规划段路由节点时延探测的对象;其次生成差分时延矩阵来保障流量调度目标时延,并设计时延矩阵的更新算法保障网络时延信息可以以分布式的方法被全网段路由节点获取并更新到自身的时延矩阵里;最后设计了段路由航点列表生成算法使段路由入口节点可以独立计算出具有指定时延保障效果的段路由航点列表。

最后,本文基于可编程的P4软件交换机bmv2对上述两种算法进行实验验证。将集中式段路由航点生成算法的对照组设置为仅用带宽计算段路由航点列表的算法,通过实验验证得到在数据中心网络和运营商网络中使用本文提出的集中式算法调度流量所产生的时延,比对照组分别降低了8.22\%和3.34\%;将分布式段路由航点生成算法的对照组设置为分布式最短路径算法,通过实验验证得到在损失一定网络吞吐量的情况下,在数据中心网络和运营商网络中用本文提出的分布式算法调度流量所产生的时延,比对照组分别降低了53.3\%和65.6\%。实验结果表明,基于时延信息进行段路由航点列表的计算可以降低流量端到端时延,为时延敏感业务提供网络层的技术支持。
}

%% 中文关键词,关键词之间用 \kwsep 分割
\ckeywords{软件定义网络 \kwsep 段路由 \kwsep 流量工程 \kwsep 网络时延}

%% 英文摘要
\eabstract{%
Segment Routing (SR) is a network layer traffic scheduling idea, the core of which is to replace the hop-by-hop destination label in the source route with a segmented route destination, this path encoding can be applied to network programming. SR has been widely used in network traffic scheduling, and as services in the network place increasing demands on latency, the network needs to consider the effects of latency guarantees at all levels, and SR techniques born at the network layer are ideal for latency guarantees.

SR policies can guide the forwarding of messages and are expressed at the packet level as a list of SR transpoints, which means any policy for SR will eventually be implemented in a list of SR transpoints. However, the current research objectives of SR transpoint list generation are often focused on optimising bandwidth usage efficiency and improving network-wide throughput, resulting in a lack of delay guarantee capability at the network layer. To address this problem, this paper proposes two SR transpoint list generation algorithms based on the architecture and design ideas of SR, from the perspective of centralised control and distributed control respectively.

First, this paper proposes the first SR transpoint generation algorithm based on the centralised control approach of SR. The algorithm further reduces the dimensionality of the data model based on the traditional topology abstraction through the global perspective of software-defined networks, with the shortest traffic end-to-end delay and the time complexity of transpoint list calculation as the main optimization objectives, and finally arrives at a SR transpoint list with delay guarantee effect.

Then, this paper proposes a second SR transpoint generation algorithm based on the distributed control method of SR. The algorithm firstly designs an adjacent SR node discovery algorithm for planning the object of node delay detection; secondly generates a differential delay matrix to guarantee the traffic scheduling target delay, and designs a delay matrix update algorithm to guarantee that the network delay information can be obtained by the network-wide SR nodes in a distributed way and updated into their own delay matrix; finally, the SR transpoint list generation algorithm is designed so that the segment Finally, the SR transpoint list generation algorithm is designed so that the SR entry nodes can independently compute a SR transpoint list with specified delay guarantee effect.

Finally, the two algorithms are experimentally validated based on the programmable P4 software switch bmv2. By setting the control group of the centralized segment route transpoint generation algorithm to the algorithm that only uses bandwidth to calculate the segment route transpoint list, the delay generated by using the centralized algorithm to schedule traffic in the data centre network and the carrier network in this paper is reduced by 8.22\% and 3.34\% respectively compared to the control group; by setting the control group of the distributed segment route transpoint generation algorithm to the distributed shortest The delay generated by the distributed route generation algorithm is reduced by 53.3\% and 65.6\% compared to the control group, respectively, in the data centre network and the carrier network with the proposed distributed algorithm for scheduling traffic at a certain loss of network throughput. The experimental results show that the calculation of SR transpoint lists based on delay information can reduce traffic end-to-end delay and provide network layer technical support for delay-sensitive services.
}

%% 英文关键词,也用 \kwsep 分割
\ekeywords{%
Software Defined Network \kwsep Segment Routing \kwsep Traffic Engineering \kwsep Network Latency}
