%%
%% This is file `example/acronyms.tex',
%% generated with the docstrip utility.
%%
%% The original source files were:
%%
%% install/buptgraduatethesis.dtx  (with options: `acronyms')
%% 
%% This file is a part of the example of BUPTGraduateThesis.
%% 

%% 定义缩略语词条的命令为
%% \newacronym{<entry>}{<abrv>}{<full_en>}{<full_zh>}
%% 其中各项的意义如下
%% <entry>    缩略语词条的名称
%% <abrv>     缩略语
%% <full_en>  英文全称
%% <full_zh>  中文名称
%% 在文中用\gls*{<entry>}等命令引用缩略语,具体用法参见用户手册

\newacronym{DFT}{DFT}
{Discrete Fourier Transform}{离散 Fourier 变换}

\newacronym{BUPT}
{BUPT}{Beijing University of Posts and Telecommunications}{北京邮电大学}

\newacronym{3GPP}{3GPP}
{The 3rd Generation Partner Project}{第三代合作计划}

\newacronym{ITU}{ITU}
{International Telecommunication Union}{国际电信联盟}

\newacronym{WTT}{WT\&T}
{Wireless Theories and Technologies Lab}{无线理论与技术实验室}

\newacronym{WTI}{WTI}
{Wireless Technology Innovation Institute}{无线新技术研究所}

\newacronym{SDN}{SDN}
{Software Defined Network}{软件定义网络}

\newacronym{MPLS}{MPLS}
{Multiprotocol Label Switching}{多协议标签交换}

\newacronym{IGP}{IGP}
{Interior gateway protocol}{内部网关协议}

\newacronym{BGP}{BGP}
{Wireless Technology Innovation Institute}{边界网关协议}

\newacronym{OSPF}{OSPF}
{Open Shortest Path First}{开放式最短路径有优先协议}

\newacronym{RSVP}{RSVP}
{Resource Reservation Protocol}{资源预留协议}

\newacronym{RSVP-TE}{RSVP-TE}
{Resource Reservation Protocol for traffic engineering}{基于流量工程扩展的资源预留协议}

\newacronym{ECMP}{ECMP}
{Equal-cost multi-path routing}{多路径等价路由}

\newacronym{SR}{SR}
{Segment Routing}{段路由}

\newacronym{SR-MPLS}{SR-MPLS}
{Segment Routing MPLS}{基于多协议标签交换转发平面的段路由}

\newacronym{SRD}{SRD}
{分段路由域}{Segment Routing Domain}

\newacronym{SLA}{SLA}
{Service-level agreement}{服务水平协议}

\newacronym{TSN}{TSN}
{Time-Sensitive Networking}{时延敏感网络}

\newacronym{INT}{INT}
{Segment Routing}{段路由}

\newacronym{OSI}{OSI}
{Open Systems Interconnection}{开放系统互连}

\newacronym{AIMD}{AIMD}
{Additive-increase/Multiplicative-decrease}{加增乘减}

\newacronym{RTT}{RTT}
{Round-trip delay}{往返时延}

\newacronym{HPCC}{HPCC}
{High Precision Congestion Control}{高精度拥塞控制}

\newacronym{TCP}{TCP}
{Transmission Control Protocol}{传输控制协议}

\newacronym{UDP}{UDP}
{User Datagram Protocol}{用户数据报协议}

\newacronym{LSP}{LSP}
{Label-switched Path}{标签交换路径}

\newacronym{IS-IS}{IS-IS}
{Intermediate system to intermediate system}{中间系统到中间系统}

\newacronym{QoS}{QoS}
{Quality of service}{服务质量}

\newacronym{LDP}{LDP}
{Label Distribution Protocol}{标签分发协议}

\newacronym{PCEP}{PCEP}
{Path Computation Element Protocol}{路径计算元素协议}

\newacronym{Ping}{Ping}
{Packet Internet Groper}{因特网包探索器}

\newacronym{P4}{P4}
{Programming Protocol-independent Packet Processorsm}{编程独立于协议的数据包处理器}

\newacronym{HTTP}{HTTP}
{Hypertext Transfer Protocol}{超文本传输协议}

\newacronym{ONOS}{ONOS}
{Open Network Operating System}{开放网络操作系统}

\newacronym{MTU}{MTU}
{Maximum Transmission Unit}{最大传输单元}

\newacronym{CSPF}{CSPF}
{Constrained Shortest Path First}{约束最短路径优先}

\newacronym{BMv2}{BMv2}
{behavioral model v2}{第二代行为模型}

\newacronym{ACK}{ACK}
{Acknowledge character}{确认字符}

\newacronym{PISA}{PISA}
{Programme for International Student Assessment}{协议独立交换机架构}

\newacronym{PPS}{PPS}
{Packets per Second}{数据包速率}

\newacronym{MCF}{MCF}
{Multi-commodity Flow}{多商品流}

\newacronym{veth}{veth}
{Virtual Ethernet}{虚拟以太网}

\newacronym{SRv6}{SRv6}
{Segment Routing IPv6}{基于IPv6的段路由}