\chapter{总结与展望}

\section{论文主要工作总结}

随着时延敏感业务的增加,网络各个方面都需要对时延需求做出保障,位于网络层的段路由可以为网络流量提供各种等级 \gls*{QoS} 的保障,因此可以使用段路由策略对网络流量的时延需求进行保障。但是现有关于段路由策略的研究,都是针对带宽或者吞吐量进行优化的,其优化目的往往是最大化端到端带宽或全网吞吐量。这是因为早期针对段路由的研究缺少及时有效的时延数据获取手段,但由于 \gls*{INT} 技术的不断发展,网络算法通过 \gls*{INT} 获取的网络时延来优化端到端时延已经卓有成效。因此本文基于 \gls*{INT} 探测结果研究了面相时延问题保障的段路由策略。

首先,本文简述了段路由的体系结构,并分析现有段路由研究的优化重点,阐释了段路由研究优化方向不够全面、无法为时延敏感流量提供针对性服务的问题,进而引出本文以时延保障为目的计算段路由策略的中心思想。并且由于段路由策略作用在数据包包头中就表现为段路由航点标签列表,因此本文聚焦于面向时延保障的段路由航点选择算法研究。本文通过调研段路由数据面和控制面的实现方案,确定集中式控制和分布式控制都可以为段路由策略计算提供足够的数据,因此本文从集中式和分布式两个方面提出段路由航点选择算法。

% 结合软件定义网络控制器阐述了集中式控制面下的段路由模型,该模型支持混合段路由节点部署的网络场景和全段路由节点的网络场景,为段路由混合网络平滑过渡提供支持。并指出通常情况下的段路由航点计算时NP-hard问题,

其次,本文提出了集中式段路由航点生成算法。该算法的优化目的是选择时延最小路径和降低控制器计算的时间复杂度。整个集中式段路由航点选择算法分为两步:第一步是在拓扑变化驱动下,对链路、节点等网络元素的属性进行计算,进而生成为了降低计算复杂度的辅助图。在这一步中为了将时延更好地结合到节点选择的算法中,本文没有直接将常用的带宽属性直接用于权重,而是将时延信息和带宽进行综合分析并构建出权重公式,并在此基础想选择中心度较高的节点构造出辅助图,辅助图的每一个节点都是备选的段路由航点;第二步是使用贝尔曼-福特算法通过在辅助图上进行有限次松弛计算得出段路由需要的航点列表,在这一步使用段路由航点列表长度作为限制条件,进一步降低了计算的复杂度。

再次,本文提出了基于差分时延分组的分布式段路由航点生成算法。该算法的优化目的是寻找最接近时延目标的段路由航点列表。其中差分时延分组是为了对每一个时延需求都可以在一定比例的范围内给予保障。基于差分时延分组的路径选择算法主要分为三步:第一步是通过主动探测的方式获取一段内邻居段路由节点列表,并在之后只针对这些节点进行探测,这是为了降低探测流量在网络中的占比;第二步是维护和更新差分时延矩阵,通过公告数据和自身探测数据对差分分组的时延矩阵进行更新,并且引入惩罚值降低段路由策略在各个分组之间的震荡;第三步是计算段路由航点列表方案,这一步主要利用差分时延矩阵可回溯特性获得整个段路由列表,与此同时增加了对列表长度进行限制的逻辑,来降低段路由报文头对网络吞吐量的消耗。

最后,本文通过实验验证了所提出两种段路由航点列表生成算法的有效性。本文基于 \gls*{P4} \cite{P4LANG} 的可编程数据面进行数据面设计,使用 \gls*{SRv6} \cite{SRARK} 技术作为算法验证的SR协议标准,设计了 \gls*{SRv6} 基础功能试验和组网流量调度实验。在基础实验中验证了搭载 \gls*{P4} 代码的 \gls*{BMv2} 软件交换机可以正确对 \gls*{SRv6} 报文进行封装、转发、解封装。在组网流量调度实验中分别验证了两种算法的有效性,其中集中式的段路由航点生成算法与只考虑链路带宽相比可以将时延结果降低8.22\%,分布式的段路由航点生成算法与最短路径分布式算法相比可以将时延结果降低65.6\%。这证明针对时延进行段路由航点列表生成的优化是非常有效的。

\section{工作展望}

本文实现了面向时延保障的段路由航点生成算法的研究,设计包括了适用于控制器的集中式算法和适用于交换机的分布式算法,验证这两种算法在流量工程中对时延保障的效果。但是纯集中式控制面或者纯分布式的控制面都有其优点和不足,因此本文今后的工作会着重集中在集中式算法和分布式算法有机结合的研究方向。从当前的研究基础来看,规模较大的拓扑会倾向于使用分布式的算法来增加网络的可扩展性,降低控制流量时延造成的影响;而复杂的网络却十分适合控制器集中控制来在错综的网络结构中找到适合服务请求的流量工程策略,因此网络拓扑的规模和类型通常是影响集中式算法和分布式算法选择取舍的主要考量点。

除了对算法本身的因拓扑而异的深入研究外,本文提出的算法在真实网络中的效果如何是一个未经验证的环节,因此在真实物理网络中验证本文算法在流量工程里应用的可行性也是日后工作的一个要点。


% 测试所有参考文献类型\cite{CITATION_BOOK,CITATION_ARTICLE,CITATION_PROCEEDINGS,CITATION_INPROCEEDINGS,CITATION_TECHREPORT,CITATION_STANDARD,CITATION_PATENT,CITATION_NEWSPAPER,CITATION_ELECTRONIC,SRSURVEYS}。

%% 本章参考文献
\ifx\usechapbib\empty
\nocite{BSTcontrol}
\setcounter{NAT@ctr}{0}
\bibliographystyle{buptgraduatethesis}
\bibliography{bare_thesis}
\fi
